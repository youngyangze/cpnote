%remake goddamn codes
\documentclass[landscape, 8pt, a4paper, oneside, twocolumn]{extarticle}
\usepackage[compact]{titlesec}
\titlespacing*{\section}
{0pt}{0px plus 1px minus 0px}{-2px plus 0px minus 0px}
\titlespacing*{\subsection}
{0pt}{0px plus 1px minus 0px}{0px plus 3px minus 3px}
\setlength{\columnseprule}{0.4pt}
\pagenumbering{arabic}
\usepackage{kotex}
\usepackage[left=0.8cm, right=0.8cm, top=2cm, bottom=0.3cm, a4paper]{geometry}
\usepackage{amsmath}
\usepackage{ulem}
\usepackage{amssymb}
\usepackage{minted}
\usepackage{color, hyperref}
\usepackage{indentfirst}
\usepackage{enumitem}
\usepackage{fancyhdr}
\usepackage{lastpage}
\pagestyle{fancy}
\fancyfoot{}

\headsep 0.2cm

\setminted{breaklines=true, tabsize=2, breaksymbolleft=}
\usemintedstyle{perldoc}

\title{Competitve Programming Note}
\author{youngyangze(Hanbyeol Lee)}
\date{Compiled on \today}

\newcommand{\revised}{Should be \textcolor{red}{\textbf{revised}}.}
\newcommand{\tested}{Should be \textcolor{red}{\textbf{tested}}.}
\newcommand{\added}{Should be \textcolor{red}{\textbf{added}}.}
\newcommand{\WIP}{\textcolor{red}{\textbf{WIP}}}
\begin{document} {
	\Large

	\maketitle

\tableofcontents
}
\thispagestyle{fancy}
\pagebreak
\textcolor{red}{\textbf{PLEASE CHECK THE STATEMENT AGAIN}}\\
\, \textcolor{blue}{\textbf{EDITOR SHOULD ADD SOME COMMENTS IN EACH THINGS}}
\section{Math}
\subsection{Exponentiation by Squaring}
{A fast exponentiation, works in $O(\log k)$}
\begin{minted}{cpp}
ll pow1(ll n, ll k) {
    ll ret = 1;
    while (k) {
        if (k & 1) ret *= n;
        n *= n;
        k >>= 1;
    }
    return ret;
}

ll pow2(ll n, ll k, ll mod) {
    if (mod == 1) return 0;
    ll ret = 1;
    n %= mod;
    while (k > 0) {
        if (k & 1) ret = (ret * n) % mod;
        k >>= 1;
        n = (n * n) % mod;
    }
    return ret;
}
\end{minted}
\subsection{Euler Phi}
{Calculates $\phi(n)$ in $O(\sqrt{n})$}\\
{Formula of $\phi(n)$: $|k\in \{1,\dots,n\}:\gcd{n,k}=1|$}
\begin{minted}{cpp}
ll phi(ll n) {
    ll ret = n;
    for (ll i = 2; i * i <= n; i++) {
        if (n % i == 0) {
            while (n % i == 0) n /= i;
            ret -= ret / i;
        }
    }
    if (1 < n) ret -= ret / n;
    return ret;
}
\end{minted}
\subsection{Xudyh's sieve}
{Finds $M(n)$(known as Mertens function) in $O(N^{\frac{2}{3}})$ where $M(n)=\sum_{i=1}^n \mu(i)$}
\begin{minted}{cpp}
const int MAX = 15000009;

bool chk[MAX + 1];
ll mu(MAX + 1, 1), pri, pre(MAX + 1, 0);

void init() {
    memset(chk, 1, sizeof(chk));
    memset(mu, 1, sizeof(mu));
    memset(pre, 0, sizeof(pre));
    mu[1] = 1;
    for (int i = 2; i <= MAX; i++) {
        if (chk[i]) {
            pri.emplace_back(i);
            mu[i] = -1;
        }
        for (int p : pri) {
            if ((ll)i * p > MAX) break;
            chk[i * p] = 0;
            if (i % p == 0) {
                mu[i * p] = 0;
                break;
            } else mu[i * p] = -mu[i];
        }
    }
    pre[0] = 0;
    for (int i = 1; i <= MAX; i++) pre[i] = pre[i - 1] + mu[i];
}

unordered_map<ll, ll> cache;
ll get(ll n) {
    if (n <= MAX) return pre[n];
    if (cache.count(n)) return cache[n];
    ll ret = 1, i = 2;
    while (i <= n) {
        ll j = n / (n / i);
        ret -= (j - i + 1) * get(n / i);
        i = j + 1;
    }
    return cache[n] = ret;
}
\end{minted}
\subsection{Berlekamp Massey}
{Requires pow2. Works in $O(N^2)$}
\begin{minted}{cpp}
const int MOD = 1e9 + 7;

vll berlekamp_massey(vll x) {
    vll ls, current;
    ll lf, ld;
    for (ll i = 0; i < x.size(); i++) {
        ll t = 0;
        for (ll j = 0; j < current.size(); j++) t = (t + 1LL * x[i - j - 1] * current[j]) % MOD;
        if ((t - x[i]) % MOD == 0) continue;
        if (current.empty()) {
            current.resize(i + 1);
            lf = i;
            ld = (t - x[i]) % MOD;
            continue;
        }
        ll k = -(x[i] - t) * _pow(ld, MOD - 2) % MOD;
        vll c(i - lf - 1);
        c.emplace_back(k);
        for (auto &j : ls) c.emplace_back(-j * k % MOD);
        if (c.size() < current.size()) c.resize(current.size());
        for (ll j = 0; j < current.size(); j++) c[j] = (c[j] + current[j]) % MOD;
        if (i - lf + (ll)ls.size() >= (ll)current.size()) tie(ls, lf, ld) = make_tuple(current, i, (t - x[i]) % MOD);
        current = c;
    }
    for (auto &i : current) i = (i % MOD + MOD) % MOD;
    return current;
}

ll get_nth(vll rec, vll dp, ll n) {
    ll m = rec.size();
    vll s(m), t(m);
    s[0] = 1;
    if (m != 1) t[1] = 1;
    else t[0] = rec[0];
    auto mult = [&rec](vll v, vll w) {
        ll m = v.size();
        vll t(2 * m);
        for (ll j = 0; j < m; j++) {
            for (ll k = 0; k < m; k++) {
                t[j + k] += 1LL * v[j] * w[k] % MOD;
                if (t[j + k] >= MOD) t[j + k] -= MOD;
            }
        }
        for (ll j = 2 * m - 1; j >= m; j--) {
            for (ll k = 1; k <= m; k++) {
                t[j - k] += 1LL * t[j] * rec[k - 1] % MOD;
                if (t[j - k] >= MOD) t[j - k] -= MOD;
            }
        }
        t.resize(m);
        return t;
    };
    while (n) {
        if (n & 1) s = mult(s, t);
        t = mult(t, t);
        n >>= 1;
    }
    ll ret = 0;
    for (ll i = 0; i < m; i++) ret += 1LL * s[i] * dp[i] % MOD;
    return ret % MOD;
}

ll guess(vll x, ll n) {
    if (n < x.size()) return x[n];
    vll v = berlekamp_massey(x);
    if (v.empty()) return 0;
    return get_nth(v, x, n);
}
\end{minted}

\subsection{FFT w/ NTT}
\added
\subsubsection{FFT?}
\added
\subsection{Kitamasa}
\WIP
\begin{minted}{cpp}
constexpr int W = 3;
using mint = modint<MOD>;
using poly = mpoly<W, MOD>;

mint kitamasa(poly c, poly a, ll n) {
    poly d = vector<mint>{1}, xn = vector<mint>{0, 1}, f;
    for (int i = 0; i < c.a.size(); i++) f.push_back(-c.a[i]);
    f.push_back(1);
    while (n) {
        if (n & 1) d = d * xn % f;
        n >>= 1;
        xn = xn * xn % f;
    }

    d.a.resize(a.size(), 0);
    mint ret = 0;
    for (int i = 0; i <= a.deg(); i++) ret += a[i] * d[i];
    return ret;
}
\end{minted}
\subsection{Mod Int}
\added
\subsection{Miller Rabin w/ Pollard Rho}
\revised
\begin{minted}{cpp}
ll mult(ll a, ll b, ll mod) {
    return a * b % mod;
}

ll power(ll base, ll exp, ll mod) {
    ll result = 1;
    while (exp > 0) {
        if (exp & 1) result = mult(result, base, mod);
        base = mult(base, base, mod);
        exp >>= 1;
    }
    return result;
}

bool miller_rabin(ll n, ll d, ll r, ll a) {
    ll x = power(a, d, n);
    if (x == 1 || x == n - 1) return true;
    for (int i = 1; i < r; i++) {
        x = mult(x, x, n);
        if (x == n - 1) return true;
    }
    return false;
}

bool is_prime(ll n) {
    if (n == 2 || n == 3) return true;
    if (n <= 1 || !(n & 1)) return false;

    ll d = n - 1, r = 0;
    while (!(d & 1)) {
        d >>= 1;
        r++;
    }

    int test[] = {2, 3, 5, 7, 11, 13, 17, 19, 23, 29, 31, 37};
    for (const int &i: test) {
        if (i > n - 2) break;
        if (!miller_rabin(n, d, r, i)) return false;
    }
    return true;
}

ll pollard_rho(ll n) {
    if (!(n & 1)) return 2;
    ll x = rand() % (n - 2) + 2, y = x, c = rand() % 10 + 1, d = 1;

    auto f = [&](ll x) { return (mult(x, x, n) + c) % n; };

    while (d == 1) {
        x = f(x);
        y = f(f(y));
        d = __gcd(abs(x - y), n);
    }
    return d;
}
\end{minted}
\subsection{Simplex}
\added
\subsection{De Brujin Sequence}
\added

\section{Geometry}
\subsection{Convex Hull}
\added
\subsection{Dynamic Convex Hull}
\added
\subsection{3D Convex Hull}
\added
\subsection{Smallest Enclosing Sphere/Circle}
\added
\subsection{K-D Tree}
\added
\subsection{Half Plane Intersection}
\added
\subsection{Delauney w/ Vornoi Diagram}
\added
\section{Data Structure}
\subsection{Segment Tree w/ Lazy}
\added
\subsection{Splay Tree}
\added
\subsection{Link Cut Tree}
\added
\subsection{Lichao Tree}
\added

\section{Strings}
\subsection{Manacher}
\added
\subsection{Hirschberg}
\added
\subsection{Round LCS}
\added
\subsection{Z}
\added
\subsection{Suffix and LCP}
\added
\subsection{NGGYU}
\revised
\begin{minted}{cpp}
We're no strangers to love
You know the rules and so do I
A full commitment's what I'm thinkin' of
You wouldn't get this from any other guy
I just wanna tell you how I'm feeling
Gotta make you understand
Never gonna give you up
Never gonna let you down
Never gonna run around and desert you
Never gonna make you cry
Never gonna say goodbye
Never gonna tell a lie and hurt you
We've known each other for so long
Your heart's been aching, but you're too shy to say it
Inside, we both know what's been going on
We know the game and we're gonna play it
And if you ask me how I'm feeling
Don't tell me you're too blind to see
Never gonna give you up
Never gonna let you down
Never gonna run around and desert you
Never gonna make you cry
Never gonna say goodbye
Never gonna tell a lie and hurt you
Never gonna give you up
Never gonna let you down
Never gonna run around and desert you
Never gonna make you cry
Never gonna say goodbye
Never gonna tell a lie and hurt you
We've known each other for so long
Your heart's been aching, but you're too shy to say it
Inside, we both know what's been going on
We know the game and we're gonna play it
I just wanna tell you how I'm feeling
Gotta make you understand
Never gonna give you up
Never gonna let you down
Never gonna run around and desert you
Never gonna make you cry
Never gonna say goodbye
Never gonna tell a lie and hurt you
Never gonna give you up
Never gonna let you down
Never gonna run around and desert you
Never gonna make you cry
Never gonna say goodbye
Never gonna tell a lie and hurt you
Never gonna give you up
Never gonna let you down
Never gonna run around and desert you
Never gonna make you cry
Never gonna say goodbye
Never gonna tell a lie and hurt you
\end{minted}
\subsection{Palindrome Tree}
\revised
\begin{minted}{cpp}
struct node {
    ll len, fail, cnt;
    vint edge;

    node(ll len) : len(len), fail(0), cnt(0), edge(26, 0) {}
};

class palindrome_tree {
private:
    vector<node> tree;
    string s;
    ll t, l;

    ll get_fail(ll x, ll pos) {
        while (pos - tree[x].len - 1 < 0 || s[pos - tree[x].len - 1] != s[pos]) x = tree[x].fail;
        return x;
    }

public:
    palindrome_tree() {
        tree.emplace_back(0);
        tree.emplace_back(-1);
        tree[0].fail = 1;
        s += '$';
        t = 1;
        l = 0;
    }

    void add_character(char c) {
        s += c;
        ll current = get_fail(l, s.size() - 1);

        if (!tree[current].edge[c - 'a']) {
            ll now = ++t;
            tree.emplace_back(tree[current].len + 2);
            tree[now].fail = tree[get_fail(tree[current].fail, s.size() - 1)].edge[c - 'a'];
            tree[current].edge[c - 'a'] = now;
        }

        l = tree[current].edge[c - 'a'];
        tree[l].cnt++;
    }

    void count_occurrences() {
        for (ll i = t; i > 0; i--) tree[tree[i].fail].cnt += tree[i].cnt;
    }

    ll max_occurrence() {
        ll ret = 0;
        for (ll i = 2; i <= t; i++) ret = max(ret, tree[i].len * tree[i].cnt);
        return ret;
    }
};
\end{minted}
\section{Graph}
\subsection{Dinic}
\added
\subsection{SCC}
\added
\subsection{2-SAT}
\added
\subsection{Directed MST}
\added
\subsection{SCC}
\added
\subsection{Global Min Cut}
\added
\subsection{General Matching}
\added
\subsection{Centroid Decomposition}
\added
\subsection{Bipartite Matching}
\added
\subsection{Dominator Tree}
\added
\subsection{MCMF}
\added
\subsection{Gomory Hu Tree}
\added
\subsection{BCC}
\added
\subsection{Block Cut Tree}
\added
\subsection{Cut Vertex/Edge}
\added

\section{Misc.}
\subsection{Basic Template}
\begin{minted}{cpp}
#include <bits/stdc++.h>

using namespace std;
using ll = long long;
using ld = long double;
using ull = unsigned long long;
using vint = vector<int>;
using matrix = vector<vint>;
using vll = vector<ll>;
using matrlx = vector<vll>;
using pii = pair<int, int>;
using pll = pair<ll, ll>;
using vpii = vector<pii>;
using vpll = vector<pll>;
using dbl = deque<bool>;
using dbltrix = deque<dbl>;
using sint = stack<int>;
using tii = tuple<int, int, int>;
using vull = vector<ull>;

#define fastio ios::sync_with_stdio(false), cin.tie(NULL), cout.tie(NULL)
#define endl '\n'
#define _CRT_SECURE_NO_WARNINGS
#define all(vec) vec.begin(), vec.end()
#define rall(vec) vec.rbegin(), vec.rend()

const int INF = 0x3f3f3f3f;
const ll VINF = 2e18;
const double PI = acos(-1);
const int MOD = 998244353;

template <typename t>
istream& operator>>(istream& in, vector<t>& vec) {
    for (auto& x : vec) in >> x;
    return in;
}

template <typename t, typename u>
istream& operator>>(istream& in, pair<t, u>& i) {
    in >> i.first >> i.second;
    return in;
}
\end{minted}
\subsection{Primes}
{We Love Primes!}\\
{List of Evil Primes}
\begin{minted}{cpp}
709, 1493, 3209, 6427, 12983, 26267, 53201, 107897, 218971
\end{minted}
{List of DFT Friendly Primes}
\begin{minted}{cpp}
65537, 4294967291, 1000000007, 7516192771, 998244353
\end{minted}
\end{document}
